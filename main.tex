\setcounter{page}{3}
\setlength{\parskip}{5pt}
\setlength{\parindent}{1,25cm}
\linespread{1.3}
\pagestyle{plain}
\fontsize{14pt}{16.8pt}\selectfont

\newpage
\begin{center}
\section*{Введение}\addcontentsline{toc}{section}{\fontsize{16pt}{16.8pt}Введение}
\label{sec:intro}
\end{center}
\par
\fontsize{14pt}{16.8pt}\selectfont
Цель курсовой работы - изучение набора макрорасширений системы компьютерной вёрстки документов TeX: LaTeX. В рамках работы необходимо изучить приниципы и способы создания текстовых документов (в т.ч. математических формул), посредством встроенных инструментов языка LaTeX.
\par
\fontsize{14pt}{16.8pt}\selectfont
Задача курсовой работы - создание документа формата .pdf, содержащего оформленный в рамках методической инструкции[1] описание упражнений дисциплины <<Алгритмизация и основы программирования>>.

\newpage
\begin{center}
\section{\fontsize{16pt}{16.8pt}Описание предметной области}
\end{center}
\par
\fontsize{14pt}{16.8pt}\selectfont
По словам официального сайта, LaTeX — это «высококачественная система набора и вёрстки» и «стандарт де-факто для обмена и публикации научных документов». С этим никто не спорит.
\par
\fontsize{14pt}{16.8pt}\selectfont
Коммерческая вёрстка книг, журналов и т. д. обычно осуществляется в настольных WYSIWYG-приложениях для подготовки публикаций, таких как InDesign, Scribus или PageMaker, разработка которого уже прекращена. LaTeX работает иначе: вы даёте ему файл, содержащий текст с кодом (т.е. с разметкой), а он выдаёт файл Postscript, который другая программа может конвертировать в PDF (некоторые варианты напрямую генерируют PDF). Если вы уделили некоторое время чтению (a) статей с конференций по информатике, (b) препринтов научных статей в свободном доступе на arXiv.org или (с) документации пакетов R, то вам знаком внешний вид этих PDF: названия отцентрированы (но не заголовки), отступ в первой строке каждого абзаца, выровненные строки, обычно широкие поля, если не используется разметка в две колонки, элегантные междусловные пробелы и всё обычно набрано этим странным, старомодным шрифтом, который называется Computer Modern.
\par
\fontsize{14pt}{16.8pt}\selectfont
Технически, LaTeX создан на базе TeX: «языка программирования специального назначения, центрального элемента системы набора и вёрстки, которая выдаёт математику (и сопровождающий текст) издательского качества» (TeX Users Group). TeX создал в конце 1970-х легендарный учёный в области информатики Дональд Кнут, разочарованный стандартами набора и вёрстки у своего издателя. Первыми пользователями стали математики, по достоинству оценившие математические символы, недоступные на печатной машинке, и красоту математических формул в новой системе. В начале 1980-х другой выдающийся учёный в области информатики Лесли Лэмпорт расширил TeX и создал LaTeX (добавив к названию первые две буквы своей фамилии). Лэмпорт добавил макросы TeX: программы, которые генерируют код TeX вместо вас, в фоне. Очень немногие пытаются писать документы напрямую в TeX. Это слишком трудно. Писать в LaTeX легче. Но это не значит, что писать в LaTeX — хорошая мысль.
\par
\fontsize{14pt}{16.8pt}\selectfont
Хотя LaTeX не так неудобен для пользователя, как TeX, написание в нём обычного текста можно считать аномалией. LaTeX — система набора и вёрстки и язык разметки. Системы набора и вёрстки обычно не используются для редактирования текста, и хотя языки разметки вроде XML и HTML часто используются таким образом, обычно это считается плохой идеей. Довольно логично утверждается, что «заставлять людей редактировать XML — это садизм» (Django Project), и хотя на протяжении какого-то времени онлайновый журнал Digital Humanities Quarterly требовал, чтобы все работы присылали в формате XML (и уникальное разнообразие XML-документов создавалось специально для этой цели), сейчас он принимает статьи и в форматах популярных текстовых редакторов. Организация Wikimedia Foundation признала требование использовать разметку wikitext при создании или редактировании статей одним из барьеров для новых пользователей. Впрочем, попытка реформы была сорвана постоянно уменьшающимся сообществом преданных волонтёров Википедии, среди которых «не выходит за рамки общепринятого мнение, что облегчение редактирования документов — это потеря времени». Я пишу этот блог в слегка упрощённой версии HTML, которая используется в текстовом редакторе WordPress, хотя каждый раз, когда эссе превышает некий размер, я думаю, что лучше бы я этого не делал. Разметка хороша для машин, для чтения и записи, но не настолько хороша для людей. Это хорошо понимали создатели текстовых редакторов вроде Microsoft Word и LibreOffice Writer, которые оба хранят текст в форме XML, но никогда не заставляют пользователя разбираться с реальным XML.
\par
\fontsize{14pt}{16.8pt}\selectfont
Несмотря на всё это, много текстов пишется именно в LaTeX. Я называю «фетишем LaTeX» убеждение, что в LaTeX есть что-то особенное, что помогает писать в нём тексты. Как мы увидим, аргументы в пользу LaTeX неубедительны на рациональном уровне: в LaTeX на самом деле довольно неудобно писать (хотя могло быть и хуже, например, это мог быть TeX). Это не значит, что от использования LaTeX следует вообще отказаться, но людям хотя бы стоит прекратить рекомендовать его в качестве текстового редактора.
\par
\fontsize{14pt}{16.8pt}\selectfont
Вот для чего хорош LaTeX: не помогать людям писать тексты, а помогать красиво оформить их.

\newpage
\begin{center}
\section{\fontsize{16pt}{16.8pt}Представление сборника задач в формате TeX }
\end{center}
\par
\fontsize{14pt}{16.8pt}\selectfont
В рамках курсовой работы была поставлена задача представления текста упражнений по предмету <<Алгоритмизация и основы программирования>> в формате TeX. Необходимо было оформить задания в формате, установленном в методическом указании[1].
\par
\fontsize{14pt}{16.8pt}\selectfont
Оформленные упражнения по предмету <<Алгоритмизация и основы программирования>>, представленны в соотвествующих подразделах.
\par

\subsection{\fontsize{14pt}{16.8pt}Упражнение 1.5}
\noindent
Вычислить по приближенной формуле значение $y=cos(x)$:\newline
$$ cos (x) \approx 1 - \frac{x^2}{2!} + \frac{x^4}{4!} - \frac{x^6}{6!} $$\newline
при $x=0,9.$ На печать вывести все слагаемые и промежуточные суммы.

\subsection{\fontsize{14pt}{16.8pt}Упражнение 2.4}
\noindent
Даны значения трёх вещественных переменных $a$, $b$ и $c$, причём два из них одинаковы. Найти значение, отличное от этих двух.

\subsection{\fontsize{14pt}{16.8pt}Упражнение 3.3}
\noindent
Вычислить значение произведения $$ y = \prod\limits_{n = 1}^1 \left( \frac{n + 1}{n!} + n \right) $$
  
\subsection{\fontsize{14pt}{16.8pt}Упражнение 4.1г}
\noindent
Протабулировать функции одной переменной:
\begin{flushright}
    $ f (x) = \sin 3x - 3 \sin x $, \tab $ 2.1375 \leq  x \leq 3.2875,\\ \triangle x = -0.25 $
\end{flushright}

\subsection{\fontsize{14pt}{16.8pt}Упражнение 5.4}
\noindent
У вектора $ X (10) $ вычислить сумму компонентов, предшествующих первому по порядку компоненту со значением из интервала $ (0, 1) $. Если такого компонента нет, то просуммировать все компоненты вектора.

\subsection{\fontsize{14pt}{16.8pt}Упражнение 6.1г}
\noindent
Вычислить с заданной абсолютной погрешностью $ABSERR$ значения элементарных функций при заданном значении аргумента $x$ :\newline
$$ arcsin x = x + \frac{x^3}{2\cdot3} + \frac{1\cdot3}{2\cdot4\cdot5}x^5 + \frac{1\cdot3\cdot5}{2\cdot4\cdot6\cdot7}x^7 + ...,$$
$$ |x| < 1; $$
    
\subsection{\fontsize{14pt}{16.8pt}Упражнение 7.1г}
\noindent
Упорядочить заданную числовую последовательность $a_1,a_2, ...,a_{100}$ так, чтобы:
$$|a_i| \geq |a_{i+1}|$$
\subsection{\fontsize{14pt}{16.8pt}Упражнение 7.9а}
\noindent
Вычислить нормы квадратной матрицы $A$, содержащей $100$ элементов $( N = 10 ):$
$$ ||A|| = \sqrt{ \sum_{i=1}^{N} \sum_{j=1}^{N} a^2_{ij} } $$
\subsection{\fontsize{14pt}{16.8pt}Упражнение 7.17г}
\noindent
В массиве $A(100,50)$ найти элемент, являющийся наименьшим по модулю.
\subsection{\fontsize{14pt}{16.8pt}Упражнение 7.28}
\noindent
В квадратной матрице $C(15,15)$ поменять местами элементы диагоналей (главной и побочной), расположенные в одной строке.
\subsection{\fontsize{14pt}{16.8pt}Упражнение 7.39}
\noindent
Сформировать одномерный массив из положительных (отрицательных) элементов двумерного массива $B(10,50)$, просматривая последний по строкам (столбцам).

\subsection{\fontsize{14pt}{16.8pt}Упражнение 8.4}
\noindent
Составить процедуру упорядочения по возрастанию элементов одномерного массива $A(M)$ и использовать её для упорядочения элементов строк заданной матрицы $C(20,15).$

\newpage
\begin{center}
\section*{Заключение}\addcontentsline{toc}{section}{\fontsize{16pt}{16.8pt}Заключение}
\label{sec:close}
\end{center}
\par
\fontsize{14pt}{16.8pt}\selectfont
В рамках курсовой работы были оформлены упражнения по дисциплине <<Алгоритмизация и основы програмирования>> в заданном формате[1].
В процессе выполнения работы был изучен синтаксис и особенности языка разметки Latex. 
\par
\fontsize{14pt}{16.8pt}\selectfont
В процессе использования подхода оформления документа с помощью языка разметки LaTeX, были выделены следующие преимущества:
\begin{spacing}{0.9}
\fontsize{14pt}{16.8pt}\selectfont
\begin{enumerate} 
  \item Легкая смена оформления документа.
  \item Переносимость результата между устройствами.
 \itemБыстрый способ набора математических формул.
\itemЛегкая смена оформления документа.
\itemПростая нумерация формул.
\itemРабота с библиографией.
\itemПлавающие объекты.
\itemПоддержка макросов.
\itemПростота генерации документов из других программ.
\itemБольшое количество дополнительных пакетов.
\end{enumerate}
\end{spacing}

\newpage
\begin{center}
\section*{Список использованных источников}\addcontentsline{toc}{section}{\fontsize{16pt}{16.8pt}Список использованных источников}
\end{center}
\par
1. Регламент содержания, оформления, организации выполнения и защиты курсовых проектов и курсовых работ / утвержден учебно-методическим советом / СПБПУ – [СПБ.,2018]. – 24 с.
\par
2. Daniel Allington, Фетиш LaTeX (или Не пишите в LaTeX! Он только для вёрстки) / [2017]
URL: https://habr.com/ru/post/339340/